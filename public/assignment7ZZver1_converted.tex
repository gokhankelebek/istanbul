
\documentclass[11pt]{article}
\usepackage{amsmath, amssymb, amsfonts}
\newcommand{\aug}{\fboxsep=-\fboxrule\!\!\!\fbox{\strut}\!\!\!}
\usepackage[margin=1in]{geometry}

\begin{document}



\begin_layout Standard
\noindent
\align center

\series bold
MA-210
\series default
: 
\emph on
Determinants and Spectral Theory
\emph default

\begin_inset space \hfill{}
\end_inset


\series bold
Assignment 7: version 1
\end_layout

\begin_layout Standard
\noindent
\align center

\series bold
Solutions by Flanders Landry
\series default

\begin_inset space \hfill{}
\end_inset


\emph on
Due by midnight on the due day
\end_layout

\begin_layout Standard
\noindent
\align left
\begin_inset VSpace 3ex*
\end_inset


\emph on
 Justify your answers by describing your all the steps taken when solving
 all the following questions.When a problem asks for the eigenvectors of
 a matrix, pick one with all integer components with the top component postive.
 
\end_layout

\begin_layout Standard

\series bold
1) 
\family roman
\series medium
\shape up
\size normal
\emph off
\bar no
\strikeout off
\xout off
\uuline off
\uwave off
\noun off
\color none
Find the determinant of the following matrix by explicitly using Gauss's
 method (showing all the steps).
 Are the matrix and the map it represents invertible?
\begin_inset Formula 
\[
H=\begin{pmatrix}1 & 1 & 3\\
1 & 0 & 1\\
1 & 1 & 2
\end{pmatrix}
\]

\end_inset


\begin_inset Newpage newpage
\end_inset


\end_layout

\begin_layout Standard

\series bold
2) 
\family roman
\series medium
\shape up
\size normal
\emph off
\bar no
\strikeout off
\xout off
\uuline off
\uwave off
\noun off
\color none
Use Gaussian's method (showing all steps) to find the determinant of this
 matrix.
\begin_inset Formula 
\[
J=\begin{pmatrix}2 & 4 & 8 & 4\\
3 & 6 & 12 & 9\\
4 & 9 & 17 & 9\\
3 & 7 & 13 & 9
\end{pmatrix}
\]

\end_inset


\begin_inset Newpage newpage
\end_inset


\end_layout

\begin_layout Standard

\series bold
3) 
\family roman
\series medium
\shape up
\size normal
\emph off
\bar no
\strikeout off
\xout off
\uuline off
\uwave off
\noun off
\color none
Consider the following diagram of two vectors.
 Write down a matrix whose determinant equals the area enclosed by the two
 vectors and the dashed lines.
 Each dotted-line-square is a square with sides of length 
\begin_inset Formula $1$
\end_inset

.
 Calculate the determinant (any way that you wish) and, in addition, calculate
 the area by appeal to Euclidean geometry.
 Display how you did this by including an annotated version of the diagram
 shown above along with the geometric derivation.
\end_layout

\begin_layout Standard
\noindent
\align center

\family roman
\series medium
\shape up
\size normal
\emph off
\bar no
\strikeout off
\xout off
\uuline off
\uwave off
\noun off
\color none
\begin_inset Graphics
	filename DetFromGraph3.png

\end_inset


\end_layout

\begin_layout Standard
\noindent
\align block

\family roman
\series medium
\shape up
\size normal
\emph off
\bar no
\strikeout off
\xout off
\uuline off
\uwave off
\noun off
\color none
\begin_inset Newpage newpage
\end_inset


\family default
\series bold
\shape default
\size default
\emph default
\bar default
\strikeout default
\xout default
\uuline default
\uwave default
\noun default
\color inherit
4) 
\family roman
\series medium
\shape up
\size normal
\emph off
\bar no
\strikeout off
\xout off
\uuline off
\uwave off
\noun off
\color none
What are the eigenvalues 
\begin_inset Formula $\lambda_{1}$
\end_inset

 and 
\begin_inset Formula $\lambda_{2}$
\end_inset

 (with 
\begin_inset Formula $\lambda_{1}<\lambda_{2}$
\end_inset

) and the corresponding eigenvectors of the following matrix? Proceed as
 follows showing and commenting on all steps: find the characteristic polynomial
 starting from the determinant of the matrix in the eigenvector equation;
 then find the characteristic polynomial; then find the factored form; then
 identify the two eigenvalues; starting from the matrix equation for each
 eigenvector, find the solution of both of those matrix equations using
 Gauss's method explicitly.
\end_layout

\begin_layout Standard
\noindent
\align block

\family roman
\series medium
\shape up
\size normal
\emph off
\bar no
\strikeout off
\xout off
\uuline off
\uwave off
\noun off
\color none
\begin_inset Formula 
\[
M=\begin{pmatrix}-1 & -6\\
-6 & 6
\end{pmatrix}
\]

\end_inset


\family default
\series default
\shape default
\size default
\emph default
\bar default
\strikeout default
\xout default
\uuline default
\uwave default
\noun default
\color inherit

\begin_inset Newpage newpage
\end_inset


\end_layout

\begin_layout Standard

\series bold
5)
\series default
 
\family roman
\series medium
\shape up
\size normal
\emph off
\bar no
\strikeout off
\xout off
\uuline off
\uwave off
\noun off
\color none
What are the eigenvalues 
\begin_inset Formula $\lambda_{1}$
\end_inset

 and 
\begin_inset Formula $\lambda_{2}$
\end_inset

 (with 
\begin_inset Formula $\lambda_{1}<\lambda_{2}$
\end_inset

) and the corresponding eigenvectors of the following matrix? Proceed as
 follows showing and commenting on all steps: find the 
\family default
\series default
\shape default
\size default
\emph on
\bar default
\strikeout default
\xout default
\uuline default
\uwave default
\noun default
\color inherit
minimal polynomial
\family roman
\series medium
\shape up
\size normal
\emph off
\bar no
\strikeout off
\xout off
\uuline off
\uwave off
\noun off
\color none
 by creating a sequence of vectors as shown in the text and class.
 Then find the factored form of the minimal polynomial; then identify the
 two eigenvalues; starting from the matrix equation for each eigenvector,
 find the solution of both of those matrix equations using Gauss's method
 explicitly.
\end_layout

\begin_layout Standard
\noindent
\align block

\family roman
\series medium
\shape up
\size normal
\emph off
\bar no
\strikeout off
\xout off
\uuline off
\uwave off
\noun off
\color none
\begin_inset Formula 
\[
N=\begin{pmatrix}2 & 4\\
4 & 8
\end{pmatrix}
\]

\end_inset


\end_layout

\begin_layout Standard
\noindent
\align block

\series bold
6) 
\family roman
\series medium
\shape up
\size normal
\emph off
\bar no
\strikeout off
\xout off
\uuline off
\uwave off
\noun off
\color none
What are the eigenvalues 
\begin_inset Formula $\lambda_{1}$
\end_inset

, 
\begin_inset Formula $\lambda_{2}$
\end_inset

 and 
\begin_inset Formula $\lambda_{3}$
\end_inset

 (with 
\begin_inset Formula $\lambda_{1}<\lambda_{2}<\lambda_{3}$
\end_inset

) and the corresponding eigenvectors of the following matrix? Show all the
 steps: Find the eigenvalues from the characteristic polynomial.
\end_layout

\begin_layout Standard
\noindent
\align block

\family roman
\series medium
\shape up
\size normal
\emph off
\bar no
\strikeout off
\xout off
\uuline off
\uwave off
\noun off
\color none
\begin_inset Formula 
\[
Q=\begin{pmatrix}\begin{array}{rrr}
1 & -2 & \hphantom{-}2\\
-2 & 1 & 1\\
2 & 2 & 4
\end{array}\end{pmatrix}
\]

\end_inset


\end_layout

\begin_layout Standard
\noindent
\align block
\begin_inset Newpage newpage
\end_inset


\end_layout

\begin_layout Standard
\noindent
\align block

\series bold
7)
\family roman
\series medium
\shape up
\size normal
\emph off
\bar no
\strikeout off
\xout off
\uuline off
\uwave off
\noun off
\color none
 What are the eigenvalues 
\begin_inset Formula $\lambda_{1}$
\end_inset

, 
\begin_inset Formula $\lambda_{2}$
\end_inset

 and 
\begin_inset Formula $\lambda_{3}$
\end_inset

 (with 
\begin_inset Formula $\lambda_{1}<\lambda_{2}<\lambda_{3}$
\end_inset

) and the corresponding eigenvectors of the following matrix? Show all the
 steps: Find the eigenvalues from the characteristic polynomial.
\end_layout

\begin_layout Standard
\noindent
\align block

\family roman
\series medium
\shape up
\size normal
\emph off
\bar no
\strikeout off
\xout off
\uuline off
\uwave off
\noun off
\color none
\begin_inset Formula 
\[
R=\begin{pmatrix}\begin{array}{rrr}
1 & 0 & \hphantom{-}0\\
0 & 1 & 0\\
1 & -1 & 1
\end{array}\end{pmatrix}
\]

\end_inset


\end_layout

\begin_layout Standard
\noindent
\align block

\series bold
8)
\family roman
\series medium
\shape up
\size normal
\emph off
\bar no
\strikeout off
\xout off
\uuline off
\uwave off
\noun off
\color none
 Find both the minimal and the characteristic polynomials.
 What are the eigenvalues 
\begin_inset Formula $\lambda_{1}$
\end_inset

, 
\begin_inset Formula $\lambda_{2}$
\end_inset

, and 
\begin_inset Formula $\lambda_{3}$
\end_inset

 (with 
\begin_inset Formula $\lambda_{1}<\lambda_{2}<\lambda_{3}$
\end_inset

) and the corresponding eigenvectors of the following matrix? 
\begin_inset Formula 
\[
G=\begin{pmatrix}\begin{array}{rrr}
1 & -1 & \quad1\\
0 & 1 & 0\\
0 & -1 & 1
\end{array}\end{pmatrix}
\]

\end_inset


\family default
\series default
\shape default
\size default
\emph default
\bar default
\strikeout default
\xout default
\uuline default
\uwave default
\noun default
\color inherit

\begin_inset Newpage newpage
\end_inset


\series bold
9) 
\family roman
\series medium
\shape up
\size normal
\emph off
\bar no
\strikeout off
\xout off
\uuline off
\uwave off
\noun off
\color none
What are the eigenvalues 
\begin_inset Formula $\lambda_{1}$
\end_inset

, 
\begin_inset Formula $\lambda_{2}$
\end_inset

, and 
\begin_inset Formula $\lambda_{3}$
\end_inset

 (with 
\begin_inset Formula $\lambda_{1}<\lambda_{2}<\lambda_{3}$
\end_inset

) and the corresponding eigenvectors of the following matrix? Find the matrices
 
\begin_inset Formula $P$
\end_inset

 and 
\begin_inset Formula $P^{-1}$
\end_inset

 which diagonalize the following matrix.
\begin_inset Formula 
\[
K=\begin{pmatrix}\begin{array}{rrr}
0 & \;1 & \;1\\
1 & 2 & 1\\
1 & 1 & 0
\end{array}\end{pmatrix}
\]

\end_inset

Using these results, find 
\begin_inset Formula $K^{n}$
\end_inset

 for all positive integers 
\begin_inset Formula $n$
\end_inset

.
 
\begin_inset space ~
\end_inset


\family default
\series default
\shape default
\size default
\emph on
\bar default
\strikeout default
\xout default
\uuline default
\uwave default
\noun default
\color inherit
Feel free to use
\family roman
\series medium
\shape up
\size normal
\emph off
\bar no
\strikeout off
\xout off
\uuline off
\uwave off
\noun off
\color none
 
\family default
\series default
\shape default
\size default
\emph on
\bar default
\strikeout default
\xout default
\uuline default
\uwave default
\noun default
\color inherit
WxMaxima, but show all the steps and describe on what is going on in each
 step.
 
\emph default

\begin_inset Newpage newpage
\end_inset


\series bold
10) 
\family roman
\series medium
\shape up
\size normal
\emph off
\bar no
\strikeout off
\xout off
\uuline off
\uwave off
\noun off
\color none
What are the eigenvalues 
\begin_inset Formula $\lambda_{1}$
\end_inset

, 
\begin_inset Formula $\lambda_{2}$
\end_inset

, 
\begin_inset Formula $\lambda_{3}$
\end_inset

 and 
\begin_inset Formula $\lambda_{4}$
\end_inset

 (with 
\begin_inset Formula $\lambda_{1}<\lambda_{2}<\lambda_{3}<\lambda_{4}$
\end_inset

) and the corresponding eigenvectors of the following matrix? Show all the
 steps: the determinant from which you obtain the characteristic polynomial,
 the characteristic polynomial, the factored form, identification of the
 eigenvalues, the matrix equation for each eigenvector, and the solution
 of the those matrix equations.
 However, use 
\family default
\series default
\shape default
\size default
\emph on
\bar default
\strikeout default
\xout default
\uuline default
\uwave default
\noun default
\color inherit
WxMaxima
\family roman
\series medium
\shape up
\size normal
\emph off
\bar no
\strikeout off
\xout off
\uuline off
\uwave off
\noun off
\color none
 to find the characteristic polynomial and also to factor it.
 Display each of the four eigenvector equations, but only solve the first
 explicitly and then use WxMaxima to solve each of the rest.
 
\begin_inset Formula 
\[
G=\begin{pmatrix}\begin{array}{rrrr}
-2 & \quad2 & \quad2 & \quad2\\
-3 & 3 & 2 & 2\\
-2 & 0 & 4 & 2\\
-1 & 0 & 0 & 5
\end{array}\end{pmatrix}
\]

\end_inset


\end_layout



\end{document}
